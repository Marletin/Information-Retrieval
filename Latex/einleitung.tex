\section{Was ist Information-Retrieval?}
Information-Retrieval (IR) beschreibt das Bereitstellen spezieller Informationen aus einer großen und unsortierten Datenmengen.
Dieses Themengebiet fällt unter Informatik, Informationswissenschaften sowie Computerlinguistik und ist ein wesentlicher Bestandteil von Suchmaschinen wie Google.
\newline
Das Thema besitzt bereits seit einigen Jahren eine hohe, aber dennoch steigende Relevanz. Die Gründe der hohen Relevanz von IR liegen vor allem beim Einsatz von Suchmaschinen. Diese sind in Zeiten des Internets die wohl wichtigste Form
der Informationsbeschaffung - und das in Bruchteilen von Sekunden. Aufgrund der immer schneller steigenden Informationsmengen wird das Thema künftig weiter an Relevanz gewinnen. Unternehmen, ebenso wie Privatanwender, wird eine immer weiter
wachsende Menge von Informationen zugänglich, die organisiert werden muss, damit relevante bzw. spezifisch gesuchte Informationen jederzeit und ohne Verzögerung gefunden werden kann.
\newline
Um das Ziel der Bereitstellung von Informationen gewährleisten zu können, werden sämtliche Informationen bzw. Dokumente, welche später gefunden werden können sollen, durchsucht und gewichtet.
Das zentrale Objekt der Informationsrückgewinnung stellt der invertierte Index dar, dessen Aufbau und Funktionsweise in den nächsten Kapiteln ausführlich erläutert wird.
Weiter wird im Verlauf dieser Arbeit die Komprimierung des Indexes sowie das Tf-idf-Maß, welches zur Beurteilung der Relevanz eines Dokumentes genutzt wird, im Fokus stehen.
\newline
Die theoretischen Hintergründe des invertierten Index, der Komprimierung und des Tf-idf-Maß werden durch eine Beispiel-Implementierung einer lokalen Suchmaschine in Programmiersprache Python veranschaulicht.

\section{Ziel der Arbeit}
Ziel der Arbeit soll es sein, ein grundlegendes Verständnis des Themenkomplexes Information-Retrieval zu vermitteln.
Darüber hinaus sollen Beispiel-Implementierungen in Python die Funktionsweise eines IR-Systemns praxisnah verdeutlichen.