\section{Problemstellung}
Wie in der Einleitung bereits angemerkt, beschreibt \glqq Information Retrieval\grqq das Bereitstellen spezieller Informationen aus einer meist großen, unsortierten Datenmenge. Dabei bekommt das System eine vom Nutzer gestellte Query (Abfrage) und versucht auf dessen Basis, Daten, die meist als Dokumente vorliegen, zurückzuliefern.
Im Gegensatz zu Abfragen im Datenbankumfeld beinhaltet die Query jedoch keinerlei Informationen, um ein spezielles Element eindeutig identifizieren zu können. Dies soll ein IR-System auch nicht leisten. Vielmehr sollen Ergebnisse zurückgeliefert werden, die mit hoher Wahrscheinlichkeit Relevanz bzgl. der gestellten Query besitzen. Der Nutzer selektiert dann die für diesen nötigen Dokumente.
\newline
Mathematisch lässt sich dies folgendermaßen formulieren:
Aus einer Dokumentenmenge $D$ soll mithilfe einer Funktion eine Teilmenge $D_1$ von $D$ ermittelt werden, die relevant für eine Abfrage $q$ ist:
\begin{defi}
	Eine Funktion $f$: $Q$ \rightarrow $D_1$ heißt Retrievalfunktion, wobei $D_1$ \subseteq $D$ gilt und $Q$ die Menge aller Queries ist.
\end{defi}